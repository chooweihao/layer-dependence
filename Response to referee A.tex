\documentclass[a4paper, 12pt]{report}
\setlength\arraycolsep{2pt}
\setlength{\parskip}{1ex plus 0.5ex minus 0.2ex}
\setcounter{tocdepth}{0}
\usepackage{graphicx}
\usepackage{chicago}
\usepackage{amsfonts}
\usepackage{pdfpages}
\bibliographystyle{chicago}


\newcommand{\eref}[1]{(\ref{#1})}
\newcommand{\fref}[1]{Figure \ref{#1}}
\newcommand{\sref}[1]{\S\ref{#1}}
\newcommand{\tref}[1]{Table \ref{#1}}
\newcommand{\aref}[1]{\ref{#1}}
\newcommand{\bi}{\begin{itemize}}
\renewcommand{\i}{\item}
\newcommand{\ei}{\end{itemize}}



\begin{document}


\section*{Response to Referee A comments}


Referee A provided a number of helpful, specific, comments on how to improve the paper and formalise ideas behind layer dependence. We have made significant revisions to our paper to address Referee A's comments. Some additional revisions were inspired by Referee A's comments.  For example we have discussed in detail several additional concepts and properties of layer dependence in the paper, e.g. section 2.


Our responses to specific comments from Referee A are: (numbering as per referee's comments)

\begin{enumerate}
\item	Agree with the potential confusion caused by the notation $(u>\alpha)$ and we have replaced it with the more standard indicator function $I_\alpha(u)$ notation.

As we are dealing mostly with random variables in the paper, the use of lower case is not likely to cause confusion with realisations of the random variables. In addition we have made clear if variables are random as opposed to constant.


\item	Agree and we have named some of the copulas illustrated in Figure 1. Remaining copulas have been specially constructed to exhibit specific dependence structures. The main aim of Figure 1 is to illustrate how layer dependence can capture a range of dependence: the actual form of the copula used is of secondary interest.
\item	Have rewritten to include Table 1 to illustrate how layer dependence changes for various copulas (independence, perfect dependence) and their transformations.
\item	Have replaced the expectations with probabilities and inserted the suggested reference.
\item	Now use indicator functions to express the tail concentration function and have also included the suggested reference.
\item	This was a typographical error and we have revised the comment.
\item	Agree and have removed the expression.
\item	Agree that many copulas can satisfy the same fitted layer dependence curve. Copula fitting based on layer dependence involves a detailed discussion and is best left to a possible separate paper. Hence have removed the discussion from the paper.
\item	Agree and we have included more discussion.
\item	Please refer to response 8. Have removed the discussion on copula fitting based on layer dependence from this paper.
\item	Please refer to response 8. Have removed the discussion on copula fitting based on layer dependence from this paper.
\item	Agree and we have amended the wording.


\end{enumerate}




\end{document}
