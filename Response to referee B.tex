\documentclass[a4paper, 12pt]{report}
\setlength\arraycolsep{2pt}
\setlength{\parskip}{1ex plus 0.5ex minus 0.2ex}
\setcounter{tocdepth}{0}
\usepackage{graphicx}
\usepackage{chicago}
\usepackage{amsfonts}
\usepackage{pdfpages}
\bibliographystyle{chicago}


\newcommand{\eref}[1]{(\ref{#1})}
\newcommand{\fref}[1]{Figure \ref{#1}}
\newcommand{\sref}[1]{\S\ref{#1}}
\newcommand{\tref}[1]{Table \ref{#1}}
\newcommand{\aref}[1]{\ref{#1}}
\newcommand{\bi}{\begin{itemize}}
\renewcommand{\i}{\item}
\newcommand{\ei}{\end{itemize}}



\begin{document}


\section*{Response to Referee B comments}


We appreciate that our original presentation has led to some of the negative reaction in the referee's  mind.   We have   streamlined the presentation so as avoid potential confusion and misunderstanding. Hence we have included additional  preliminary detail and discussion to motivate layer dependence.
Our responses to Referee B's specific comments are: (in referee's order)


\begin{enumerate}
\item Included additional explanation of how layer dependence captures the dependence structure for each copula illustrated in Figure 1, i.e. by reflecting the scatter of points continuously along the 45 degree line.
\item We have removed the factor copula model discussion and fitting strategy, focussing instead on layer dependence and its use as a local dependence measure.
\item Regarding the third comment, we agree and we have used the more common notation $I_\alpha(u)$ for indicator variables.



\end{enumerate}




\end{document}
